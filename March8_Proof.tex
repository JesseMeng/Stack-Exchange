\documentclass{article}
\usepackage[utf8]{inputenc}

\title{Stack Exchange Answers}
\author{Jesse Meng }
\date{March 8, 2018}

\begin{document}

\maketitle

\section{Original Post}
Title : How to prove an implication within an if and only if

Suppose you need to prove that $A\iff (B\implies C)$.

The two ways to prove this are:

---

(1a): Suppose $A$ and $B$ are true. Prove that $C$ is true.

(1b): Suppose $B$ and $C$ are true. Prove that $A$ is true.

---

(2a): Suppose $A$ and $B$ are true. Prove that $C$ is true.

(2B): Suppose $A$ is not true and B is true, prove that $C$ cannot be true.

---

Are these ways correct? I always get confused what you can assume and what you have to prove when there's multiple implications and such in one statement.
\section{My Solution}
Prove both directions:

Forward: assume A, prove ($B \implies $C). This means "suppose A and B are true. Prove that C is true".

Reverse: Assume ($B \implies $C), prove A. So, you don't  "Suppose B and C are true. Prove that A is true". Instead, "suppose the truth of B implies truth of C, then prove that A is true. **$1b$ is incorrect**.
Now, reverse can also be interpreted as suppose A is not true, prove ($B \implies $C) is not true by the contrapostive. Which means prove $B$ is true and $C$ is false. So you don't "Suppose A is not true and B is true, prove that C cannot be true." Instead, "Suppose A is not true, prove B is true and prove that C cannot be true." **$2b$ is incorrect.**
\end{document}