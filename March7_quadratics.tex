\documentclass{article}
\usepackage[utf8]{inputenc}

\title{Stack Exchange Answers}
\author{Jesse Meng }
\date{March 7, 2018}

\begin{document}

\maketitle

\section{Original Post}
Title : Advantages and Disadvantages of the different forms of a quadratic function

My question is when sketching a graph what are the advantages and disadvantages for the following:

i. Standard form

ii. Factorised form 

iii. Vertex form 

Many thanks!

\section{My Solution}
- **Factorized Form**

With factorized form, you can easily see the two roots of the quadratic, which means you can sketch the shape of the graph easily by plotting the two zeros. Then draw an upward curve or downward curve depending on the sign on the $x^2$ term. The x-coordinate of the vertex lies exactly half way in between the roots. And hence you can determine the y-coordinate by substitution.

 - **Vertex Form**

With vertex form, you can easily graph the vertex point and the general shape of the quadratic given the leading coefficient. However, you cannot determine the zeros immediately.

 - **Standard Form**

For standard form, you will only know whether the quadratic is concave up or down and factorized/vertex forms are better in terms of sketching a graph. However, there are other advantages involved with the standard form, such as the easiness of computing the derivative and then using it to find the vertex. 

- **Conclusion**

So, from the analysis, factorized form seems to be the best choice for graphing, as the roots are trivial to find and vertex can be computed easily. Whereas the vertex form requires a little more effort to solve for the zeroes of the quadratic.
\end{document}