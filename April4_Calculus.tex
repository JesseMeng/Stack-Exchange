\documentclass{article}
\usepackage[utf8]{inputenc}

\title{Stack Exchange Answers}
\author{Jesse Meng }
\date{April 4, 2018}

\begin{document}

\maketitle

\section{Original Post}
Title : Given the curve $y = 2 x ^3 + 3 x ^2 − 36 x$ on the interval $− 1 \le x \le4$ , find the absolute minimum?

I know I have to find the first derivative which is $y'=6x^2+6x-36.$
Then I set it to $0.$ Which I got $0 = 6(x^2+x-6)$ and then $x=-3$ or $x=2.$
\section{My Solution}
Up until you have Absolute minimum occurs either at the boundary points or at the points where the derivative equals to zero or does not exist. So, you've already found $x=2$ is a point in the interval such that the derivative is $0$. You would only need to check the function's value at $x=2$ and the boundary points $x=-1,x=4$ to determine the absolute minimum point in that interval.
\end{document}