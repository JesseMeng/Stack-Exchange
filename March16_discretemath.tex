\documentclass{article}
\usepackage[utf8]{inputenc}

\title{Stack Exchange Answers}
\author{Jesse Meng }
\date{March 16, 2018}

\begin{document}

\maketitle

\section{Original Post}
Title : Counting: passwords

Sorry if this has been beaten to death. 

A password can contain capital and lowercase letters, digits, and underscores.

$63^8$ total passwords, correct?


If we add the stipulation that the first character cannot be a digit-- why wouldn't the number of illegal passwords be $10*63^7$, resulting in $63^8-10*63^7$ legal passwords? 
\section{My Solution}
Your logic is correct for 8 digit passwords. For each digit, you could have $26*2+10+1=63$ choices. So, without restrictions,total number of passwords would be $63^d$ where d is the number of characters in the password. If the first character cannot be a digit, then we would only have $53$ choices for the first character, which then yields $53*63^{d-1}$ choices.
\end{document}