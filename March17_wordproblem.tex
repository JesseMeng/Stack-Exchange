\documentclass{article}
\usepackage[utf8]{inputenc}

\title{Stack Exchange Answers}
\author{Jesse Meng }
\date{March 17, 2018}

\begin{document}

\maketitle

\section{Original Post}
Title : A sweet water contains $\%40$ of sugar. $\%20$ of this water is poured. Then, pure sugar is being added as much as poured water.

A sweet water contains $40\%$ of sugar. $20\%$ of this sweet water is poured. Then, the pure sugar is being added as much as poured water. What is the percentage of sugar in the new mixture?

Is there any formula/strategy to solve mixture problems? These questions are making me confused. That's why I couldn't think anything about it. 

Regards,
\section{My Solution}
In this type of questions, if you want to have an easier time, you could try to assign a value to the total weight, say $100g$. Then, you have $40g$ of sugar. After pouring $20g$ of water, you only have $32g$ of sugar left since the pourred water had $8g$ of sugar. Then, you add in $20g$ of sugar, now there is $52g$ of sugar. So, your percentage is $52$%. This method works because we only care about the ratio of sugar to the water. So we can give arbitrary values to the actual amount.
\end{document}