\documentclass{article}
\usepackage[utf8]{inputenc}

\title{Stack Exchange Answers}
\author{Jesse Meng }
\date{March 3, 2018}

\begin{document}

\maketitle

\section{Original Post}
Title : When does the inequality change in probability problems and why does it?

From what I've learned in my stats class when you have $P(X \gt x)$ you make it $P(X \gt x) = 1-P(X \le x)$. I think I understand that you have to do this inequality change because you can't calculate a probability of any possible number being greater than $x$, but I don't understand why greater than changes to less than *or* equal to. Why is this the case?

Also, does it work that way in reverse? Is $P(X \ge x)=1-P(X \lt x)$? 
\section{My Solution}
This equation is based on the fact that the sum of the probabilities for all possible values of $x$ is $1$ and that values of $x$ have order imposed on them. Then it is easy to see that $$ \sum P(x)=1 $$ and that given any given any $x$, you can divide the set of all possible values of $x$ into $3$ disjoint sets, the set which has $ S_1 = \{ X | X < x \}$, $S_2 = \{x\}$ and $S_3 = \{X | X > x\}.$ So $P(S_1)+P(S_2)+P(S_3) = 1$.

Consequently $ P(S_3) = P(X > x) = 1-P(S_1)-P(S_2) = 1- P(X<x)-P(X=x) = 1-P(X\leq x)$
From, this, it is also easy to see that if the same assumptions are met, the reverse can also be done.
\end{document}