\documentclass{article}
\usepackage[utf8]{inputenc}

\title{Stack Exchange Answers}
\author{Jesse Meng }
\date{March 6, 2018}

\begin{document}

\maketitle

\section{Original Post}
Title : Identity matrix in adjoint proof

>Prove: If A is invertible, then adj(A) is invertible and $[adj(A)]^{-1}=\frac{1}{det(A)}A$

It starts using the adjoint rule

$$A^{-1}=\frac{1}{det(A)}adj(A)$$

$$AA^{-1}=\frac{1}{det(A)}A*adj(A)\longrightarrow I=\frac{1}{det(A)}A*adj(A)$$



and the proof goes on to say, 

$$[adj(A)]^{-1}=\frac{1}{det(A)}A$$

But I'm not sure how to show (on the left side): $$\frac{I}{[adj(A)]}=[adj(A)]^{-1}$$
Why can $I$ be used interchangeably with $1$?


\end{document}