\documentclass{article}
\usepackage[utf8]{inputenc}

\title{Stack Exchange Answers}
\author{Jesse Meng }
\date{April 9, 2018}

\begin{document}

\maketitle

\section{Original Post}
Title : Simplify $\ln(e^{2x+1})$

I was trying to do an integration problem with u sub and got stuck, one part of the equation was this
$\ln(e^{2x+1})$

this is suppose to simplify really nicely according to a site, is there a rule for this i get that $\ln(e) =1$ and that $\ln(e^{x})=x$. So would $\ln(e^{2x+1})=2x+1$? Is this a rule that I may have forgotten about?
I'm more interested in the properties behind this if it is a rule, just out of curiosity as to why it is so.
\section{My Solution}
Think it this way: $\ln(e^{f(x)})$ defines the exponent such that if you take $e$ to the power of that exponent, will equal to $f(x)$. So, what power does $e$ need to be raised to for it to equal to $e$ to the power of $f(x)$? $f(x)$ itself, of course, and this works for any exponent since we made no assumptions on it.
\end{document}
