\documentclass{article}
\usepackage[utf8]{inputenc}

\title{Stack Exchange Answers}
\author{Jesse Meng }
\date{April 7, 2018}

\begin{document}

\maketitle

\section{Original Post}
Title : Evaluate $ \lim_{x\to\infty}\frac{\ln(1+e^x)}{x}$

I have intuitively guessed that the answer should be 1 as $x\to +\infty$. How do I formally prove this? Also how do I evaluate the limit as $x\to -\infty $.

UPDATE: Based on the answers below, I have managed to do the following.
$$
\lim_{x\to\infty} \frac{\ln (1+e^x)}{x} \\
= \lim_{x\to\infty} \frac{\ln e^x(\frac{1}{e^x}+1)}{x} \\
= \lim_{x\to\infty} \frac{x +\ln (\frac{1}{e^x}+1)}{x} \\
= \lim_{x\to\infty} 1+ \ln (1+\frac{1}{e^x})^\frac{1}{x} \\
$$
How do I proceed proceed from here?
\section{My Solution}
Since you would get $\frac{\infty}{\infty}$

You can use l'hopital's rule:

 $ \lim_{x\to\infty} \frac{\ln(1+e^x)}{x}=\frac{e^x}{1+e^x}=\frac{1}{e^{-x}+1}=1$

Formal proof of this relies on the proof of this case for l'hopital's rule.

Add on: for $-\infty$, solution should be trivial, treat $-\infty$ as an extended real number and substitute.
\end{document}