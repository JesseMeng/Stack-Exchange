\documentclass{article}
\usepackage[utf8]{inputenc}

\title{Stack Exchange Answers}
\author{Jesse Meng }
\date{March 8, 2018}

\begin{document}

\maketitle

\section{Original Post}
I am bit confused on the application of the logarithm rules when it comes to using them to determine the order of growth. 

For example:

1. $ 2^{\log 2n} + 4n = \Theta(2^n) $ 

2. $ 2^{2\log 2n} + 4n = \Theta(n^2) $ 

3. $ n\log n + 10n^2 + 5^{\log n} = \Theta(n^{\log 5}) $ 

4. $ n^{\log n} + 4^{(\log n)^2} = \Theta(n^{\log n^2}) $ 

Which logarithm rules apply to the above? How exactly are they solved? And how can I determine the order of growth for these types of questions when doing them in the future?

NB: The log is in base 2

 
\section{My Solution}
Your first rule is incorrect:

For example, the logarithms in computer science are in base $2$.
So:

1. $ 2^{\log 2n} + 4n = \Theta(2^n) $ can be simplified to $$6n= \Theta(2^n)$$ which is false as $6n$ grows slower.

The second rule is proven as follows:
$$2^{2 \log_2(2n)}=\left(2^{\log_2(2n)^2}\right)=(2n)^2=4n^2=\Theta(n^2)$$

Proof of third rule:

$n\log(n)$ is insignificant compared to $n^2$.
$$5^{\log n}=5^\frac{\log_5(n)}{\log_5(2)}=n^\frac{1}{\log_5(2)}=n^{\log_2(5)}$$
$n^2$ is insignificant compared to $n^{\log(5)}$. Hence we have arrived at our conclusion.

Proof of fourth rule:
$$4^{(\log n)^2}=2^{2{(\log n)^2}}=((2^{\log n})^{logn})^2=n^{(\log n^2)}$$
$n^{\log(n)}$ is insignificant compared to $n^{(\log n^2)}$, we have arrived at our conclusion.

Note that these rules all proven with the basic properties of $\log$, such as converting bases. You should sharpen on those fundamentals.
\end{document}