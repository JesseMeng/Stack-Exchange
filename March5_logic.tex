\documentclass{article}
\usepackage[utf8]{inputenc}

\title{Stack Exchange Answers}
\author{Jesse Meng }
\date{March 5, 2018}

\begin{document}

\maketitle

\section{Original Post}
Title : Logic proof verification - Hoare

I'm working on a Hoare logic proof but am having some difficulty:  Here is the problem:

    All variables are real:
    {TRUE}
        if a<b then
        c := a;
        a := b;
        b := c
    end if
    	{b <= a}


I think that I have to start with the assignment rule.  So perhaps something like this to start:

    {c<=a}b := c {b<=a} -- by the assignment rule

I'm pretty confused by Hoare logic in general though.  I can provide a complete example though:

        {x > 0}
         i := j^2
         n := sqrt(i)
        {n > 0}
   
        1. { √i > 0} n:=√i {n > 0} (Assignment rule)
        2. i > 0 →  √i > 0 [Side condition]
        3. {j^2 > 0} i:=j^2 {i > 0} (Assignment rule)
        4. {j^2 > 0} i:=j^2 {√i > 0} (e)2,3 (weaken post condition using lines 2 and 3)
        5. {j^2 > 0} i:=j^2 ; n:=√i {n > 0} (Compound statement rule using line 4 and 1)
        6. j > 0 → j^2 > 0 [Side condition]
        7. {j > 0} i:=j^2 ; n:=√i {n > 0} (i)6,5 (strengthen precondition using lines 6 and 5)
\section{My Solution}
Yes, you'll have to show that if the if condition does not hold, then you have the logical conclusion(which follows trivially in this case).
Next, for the if statements. you'll have to work backwards. Your first step is correct, start from $ b\leq a  $, when you see $b=c$, replace b by c to have $ c\leq a  $, then you have the same steps for the next two assignments as you go upwards from $ c\leq a  $. When you are finished moving through the assignments, you'll then need to show that $a<b$ implies what you have previously from moving up $3$ assignments.
\end{document}