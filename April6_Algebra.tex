\documentclass{article}
\usepackage[utf8]{inputenc}

\title{Stack Exchange Answers}
\author{Jesse Meng }
\date{April 6, 2018}

\begin{document}

\maketitle

\section{Original Post}
Title : Solving equations by taking square roots

I am confused with solving equations by taking square roots. I have a math problem below as an example of what I am confused with.

$12x^2-154=0$

So first I believe you are supposed to move the constant to one side so it would be 

$12x^2=154 $

After that I believe we need to divide by $12$, however $154$ is not divisible by $12$ nor do any perfect squares fit. I am aware they both share the greatest common factor of $2$, however I am not sure what to do with that or if it is needed. I am not sure.
\section{My Solution}
You can work with fractions, I assume?
Of course, in this case, your solutions for $x$ cannot be of integers, as you have already noticed. However, you can always do:
$$12x^2=154$$
$$x^2=\frac{77}{6}$$
$$x=\pm\sqrt\frac{77}{6}$$
\end{document}