\documentclass{article}
\usepackage[utf8]{inputenc}

\title{Stack Exchange Answers}
\author{Jesse Meng }
\date{March 8, 2018}

\begin{document}

\maketitle

\section{Original Post}
Title : Finding smallest nonnegative integer x in a modulo equation

-36789 = x mod 19

So what I have done is the following

1.  36789/19 = 1936.263158.... 
             = 1937 (round up)

2.  (1937 * 19) - 36789 = 14 <- assumed final answer

But I am unsure if the final answer is correct, as I am new to this modulo arithmetic.

I have read online modulo arithmetic, but unable to find a scenario like the above question.

The typical example is -36789 mod 19 = 14 (which the assumed final answer)... 
\section{My Solution}
You are looking for the smallest non-negative value for $-36789+19n$, because that is equivalent to the $-36789$ $mod$ $19$. Therefore, it is clear that when $n<1937$, $-36789+19n <0$, when $n>1937$, $-36789+19n >0$ and $-36789+19n$  is strictly increasing. Hence, $1937$ is the value for n that will create the smallest non-negative value for $-36789+19n$. Consequently, $-36789+19*1937=14$.
\end{document}
